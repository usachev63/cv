\documentclass[11pt,a4paper]{moderncv} 
\usepackage[utf8]{inputenc}  
\usepackage[russian]{babel}
\usepackage[scale=0.75]{geometry}
\recomputelengths

\renewcommand{\familydefault}{\sfdefault}

\moderncvstyle{classic}                             % style options are 'casual' (default), 'classic', 'banking', 'oldstyle' and 'fancy'
\moderncvcolor{blue}                               % color options 'black', 'blue' (default), 'burgundy', 'green', 'grey', 'orange', 'purple' and 'red'

\name{Данила}{Усачев}
\title{Curriculum Vitae}
\email{duesna897@gmail.com}
\social[github]{usachev63}
\extrainfo{
 Telegram: \href{https://t.me/usachev63}{@usachev63} \\
 \url{https://t.me/usachevtech}
}

\begin{document}
\maketitle

\section{Компетенции}

\cvlistitem{Системное программирование на C++}
\cvlistitem{Статический анализ программ}

\section{Опыт работы}

\cventry{2025--2026}{Kaspersky Lab (0.5 года)}{}{}{}{
 С++-разработчик в Kaspersky Thin Client --- решение для тонких клиентов на базе KasperskyOS
}

\cventry{2022--2025}{Huawei R\&D (3 года)}{Исследовательский центр им. Чебышева, лаборатория <<разработки и верификации сетевого ПО>>}{Санкт-Петербург, Россия}{}{
 Разрабатывал и поддерживал предметно-ориентированный инструмент статического анализа кода, написанный на C++ и основанный на Clang, предназначенный для анализа и тестирования крупной кодовой базы сетевого ПО на языке C. \\
 Ключевые высоко оценённые достижения:
 \begin{itemize}
  \item Разработал специфичный вид статического анализа, автоматический выявляющий массивы в C-программах и аппроксимирующий их длину и предназначенный для задач фаззинга. В результате покрытие кодовой базы увеличилось на 10\%, и это позволило выявить множество программных ошибок (число зарегистрированных ошибок за определённый период увеличилось на 40\%). См.~\url{https://t.me/usachevtech/4}
  \item Оптимизировал критическую часть пайплайна тестирования в 300-400 раз, создав высокопроизводительный инструмент на C++
 \end{itemize}
}

\section{Спортивное программирование}
Начиная с 2017 (7 класс в школе) у меня был длинный и успешный опыт в спортивном программировании, с несколькими всероссийскими и международными победами. См. мой \href{https://codeforces.com/profile/usachevd0}{\color{blue}профиль на Codeforces}. Мои ключевые достижения:

\bigskip

\cvlistitem{\textbf{Серебряная медаль}, 33-я международная олимпиада по информатике (IOI), 2021.}
\cvlistitem{\textbf{Золотая медаль}, 8-я международная олимпиада RMI, 2020.}
\cvlistitem{\textbf{Золотая медаль}, 12-я международная олимпиада IATI, 2020.}
\cvlistitem{\textbf{Призёр} 33-й всероссийской олимпиады школьников по информатике, 2021, Москва, Россия.}

\newpage
\section{Образование}

\cventry{2021--2025}{Бакалавриат <<Современное программирование>>}{Санкт-Петербургский государственный университет, факультет математики и компьютерных наук}{}{}{Дипломная работа \textit{<<Автоматический поиск массивов в C-программах с помощью статического анализа>>} \url{https://t.me/usachevtech/4}\\ Спец. курсы: языки программирования и виртуальные машины, разработка компиляторов, управление памятью, анализ кода и безопасная разработка, тестирование ПО, параллельное программирование, язык Go, язык Scala, теория графов, операционные системы, базы данных, архитектура компьютера, вычисления на видеокартах}

\section{Пет-проекты}

\cvlistitem{\textbf{SAKLS} (\emph{Syntax-Aware Keyboard Layout Switching}) --- проект для автоматического переключения раскладки клавиатуры в текстовом редакторе, в зависимости от синтаксиса. Доступен как плагин \href{https://github.com/sharkov63/sakls.nvim}{\color{blue}sakls.nvim} для редактора Neovim, и как библиотека-бэкенд \href{https://github.com/sharkov63/sakls}{\color{blue}sakls}.}

\end{document}
