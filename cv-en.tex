\documentclass[11pt,a4paper]{moderncv} 
\usepackage[utf8]{inputenc}  
\usepackage[scale=0.75]{geometry}
\recomputelengths

\renewcommand{\familydefault}{\sfdefault}

\moderncvstyle{classic}                             % style options are 'casual' (default), 'classic', 'banking', 'oldstyle' and 'fancy'
\moderncvcolor{blue}                               % color options 'black', 'blue' (default), 'burgundy', 'green', 'grey', 'orange', 'purple' and 'red'

\name{Danila}{Usachev}
\title{Curriculum Vitae}
\email{duesna897@gmail.com}
\social[github]{usachev63}
\extrainfo{
 Telegram: \href{https://t.me/usachev63}{@usachev63} \\
 \url{https://t.me/usachevtech}
}

\begin{document}
\maketitle

\section{Expertise}

\cvlistitem{C++ system programming}
\cvlistitem{Static program analysis}

\section{Professional Experience}

\cventry{2025--2026}{Kaspersky Lab (0.5 years)}{}{}{}{
 C++ SWE at Kaspersky Thin Client --- KasperskyOS based solution for thin clients
}

\cventry{2022--2025}{Huawei R\&D (3 years)}{Chebyshev Research Center, Network Software Engineering and Verification Laboratory}{Saint-Petersburg, Russia}{}{
  Developed and maintained a domain specific static analysis tool written in C++ and based on Clang, to analyze and test large scale codebase of network devices software in C language. \\ Key achievements:
  \begin{itemize}
   \item Developed a specific kind of static analysis, which automatically finds arrays in C programs and approximates their length, designed for fuzzing pipelines. As a result, codebase coverage increased by 10\%, and this allowed to find many errors (number of registered errors over a period of time increased by 40\%). \\See~\url{https://t.me/usachevtech/4}
   \item Improved performance of a critical part of testing pipeline by 300-400 times by developing a high-performance tool in C++
   % \item Implemented several analyses based on Steensgaard-style points-to analysis engine, the output of which is used as guidance for fuzzing and improved the test generation pipeline performance and quality.
   % \item Authored an approach, which uses LTO, Whole Program assumption and source-code level instrumentation, with which we were able to reduce size of shared libraries by removing unused code.
  \end{itemize}
}

\section{Competitive Programming}
Starting from 2017 (7th year at school) I've had a long and successful experience in competitive programming, with multiple national and international achievements. See my \href{https://codeforces.com/profile/usachevd0}{\color{blue} \textit{Codeforces page}}. My best achievements are:

\cvlistitem{\textbf{Silver medal}, the 33rd International Olympiad in Informatics (IOI), 2021.}
\cvlistitem{\textbf{Gold medal}, the 8th Romanian Master of Informatics (RMI), 2020.}
\cvlistitem{\textbf{Gold medal}, the 12th International Autumn Tournament of Informatics (IATI), 2020.}
\cvlistitem{ \textbf{Winner} of the 33rd Russian National
Olympiad in Informatics, 2021, Moscow, Russia.}

\newpage
\section{Education}

\cventry{2021--2025}{Bachelor, ``Modern Software Engineering''}{St. Petersburg State University, Department of Mathematics and Computer Science}{}{}{
 Thesis: ``Automatic array finding in C programs using static analysis'' \url{https://t.me/usachevtech/4} \\
 Courses: programming languages and virtual machines, compiler design, memory management, application security, software testing, parallel computing, Go language, Scala language, graph theory, operation systems, databases, computer architecture, GPU computing
}

\section{Pet-projects}

\cvlistitem{\textbf{SAKLS}: \emph{Syntax-Aware Keyboard Layout Switching} project for automatically switching layout in a text editor. Available as \href{https://github.com/sharkov63/sakls.nvim}{\color{blue}sakls.nvim} Neovim plugin and \href{https://github.com/sharkov63/sakls}{\color{blue}sakls} backend library.}

\end{document}
